\documentclass[dvipsnames]{beamer}
\usetheme{ru}

\usepackage{minted}
\usepackage{mathspec}
\usepackage{xcolor}

\usepackage{tikz}
\usetikzlibrary{calc,shapes.multipart,chains,arrows}

\setmainfont[Ligatures=TeX]{XITS}
\setmathfont{XITS Math}

\title{agdARGS}
\subtitle{Declarative Hierarchical Command Line Interfaces}
\author[G. Allais]{Guillaume Allais}
\institute[Radboud]{Radboud University}
\date[TTT 17]{Type Theory based Tools 2017 - Paris}

\begin{document}

\begin{frame}
 \maketitle
\end{frame}

\section{Motivation: When it typechecks...}

\begin{frame}{Ship it? More like shelve it! :(}
  Core algorithm
    \begin{itemize}
      \item Data structures with strong invariants
      \item Fully certified
    \end{itemize}
  + Boilerplate
    \begin{itemize}
      \item Validation of unsafe data
      \item (Command Line / Graphical) Interface
    \end{itemize}
  = Executable
\end{frame}

\begin{frame}{Case study: a minimal \texttt{grep}}
No access to the program's command-line arguments
  \begin{itemize}
    \item Add postulate + \texttt{COMPILED} pragma for \texttt{getArgs}
    \item Wrap in the \texttt{IO} monad
  \end{itemize}
Ad-hoc parsing function
\end{frame}

\begin{frame}[fragile]{"Hand-crafted" solution}

Now that we have access to the arguments, we just have to make sense
of them. We use a type of options:

\begin{minted}{haskell}
record grepOptions : Set where
  field
    -v     : Bool            -- invert match
    -i     : Bool            -- ignore case
    regexp : Maybe String    -- regular expression
    files  : List FilePath   -- list of files to process
\end{minted}

And "hand-craft" a function populating it:
\end{frame}

\begin{frame}[fragile]
\begin{minted}{haskell}
parseOptions : List String -> grepOptions
parseOptions args =
  record result { files = reverse (files result) }
  where
    cons : grepOptions -> String -> grepOptions
    cons opt "-v" = record opt { -v = true }
    cons opt "-i" = record opt { -i = true }
    cons opt str  =
      if is-nothing (regexp opt)
      then record opt { regexp = just str }
      else record opt { files  = str :: files opt }

    result : grepOptions
    result = foldl cons defaultGrepOptions args
\end{minted}
\end{frame}

\section{What is the specification of a CLI?}
\begin{frame}{Types: What is a command-line interface?}

What is a Command-Line Interface?
  \begin{itemize}
    \item A \textbf{description}
    \item A list of \textbf{subcommands}
    \item A list of \textbf{modifiers} (\textbf{flags} \& \textbf{options})
    \item Default \textbf{arguments}
  \end{itemize}

What should we get from declaring one?
  \begin{itemize}
    \item The corresponding parser
    \item Usage information
  \end{itemize}
\end{frame}

\begin{frame}[fragile]{Types: Example of a CLI}
\begin{minted}{haskell}
Grep = record
  { description = "Print lines matching a regexp"
  ; subcommands = noSubCommands
  ; arguments   = lotsOf filePath
  ; modifiers   = 
    , "-v" ::= flag "Invert match"
    < "-i" ::= flag "Ignore case"
    < "-e" ::= option "Regexp" regexp
    < <>
  }
\end{minted}
\end{frame}

\section{Internal representation}
\begin{frame}{Extensible records}
 Represent field names as sorted lists
  \begin{itemize}
    \item guaranteed uniqueness of commands / modifiers
    \item easy to lookup values
    \item easy to extend
    \item first class citizens (generic programming possible!)
  \end{itemize}
 Associate a type to each field name

 Generate record types by recursion on the list of field names

 Remark: Drive type inference
\end{frame}

\subsection{Types - Keep your neighbours in order}

\begin{frame}{The type of extensible records}
  McBride to the rescue: "How to keep your neighbours in order" tells
  us how to build in the invariant stating that a tree's leaves are
  sorted.

  In the special case of linked lists, using a \emph{strict} total
  order, we move from:
%%%%%%%%%%%%%%%%%%%%%%%%%%%%%%%%%%%%%%%%%%%%%%%%%%%%%%%%%%%%
%%%%%%%%%%%%%%%%%%%%%%%%%%%%%%%%%%%%%%%%%%%%%%%%%%%%%%%%%%%%
\begin{figure}[t]
\centering
\begin{tikzpicture}[list/.style={rectangle split, rectangle split parts=2,
    draw, rectangle split horizontal}, >=stealth, start chain]

  \node[list,on chain] (A) {12};
  \node[list,on chain] (B) {99};
  \node[list,on chain] (C) {128};
  \node[on chain,draw, inner sep=6pt] (D) {};
  \draw (D.north west) -- (D.south east);
  \draw (D.north east) -- (D.south west);
  \draw[*->] let \p1 = (A.two), \p2 = (A.center) in (\x1,\y2) -- (B);
  \draw[*->] let \p1 = (B.two), \p2 = (B.center) in (\x1,\y2) -- (C);
  \draw[*->] let \p1 = (C.two), \p2 = (C.center) in (\x1,\y2) -- (D);
\end{tikzpicture}
\end{figure}

  To the proven ordered:

\begin{figure}[t]
\begin{tikzpicture}[list/.style={rectangle split, rectangle split parts=2,
    draw, rectangle split horizontal}, >=stealth, start chain]

  \node[list,on chain] (A) {{\color{red}-$\infty$} < 12};
  \node[list,on chain] (B) {{\color{gray}12} < 99};
  \node[list,on chain] (C) {{\color{gray}99} < 128};
  \node[on chain,draw] (D) {{\color{gray}128} < {\color{red}+$\infty$}};
  \draw[*->] let \p1 = (A.two), \p2 = (A.center) in (\x1,\y2) -- (B);
  \draw[*->] let \p1 = (B.two), \p2 = (B.center) in (\x1,\y2) -- (C);
  \draw[*->] let \p1 = (C.two), \p2 = (C.center) in (\x1,\y2) -- (D);
\end{tikzpicture}
\end{figure}

%%%%%%%%%%%%%%%%%%%%%%%%%%%%%%%%%%%%%%%%%%%%%%%%%%%%%%%%%%%%%%%%%%%%
%%%%%%%%%%%%%%%%%%%%%%%%%%%%%%%%%%%%%%%%%%%%%%%%%%%%%%%%%%%%%%%%%%%%
\end{frame}

\begin{frame}[fragile]{Key ideas}

  Extend any ordered set with +/-infinity:
  \begin{minted}{haskell}
  data [_] (A : Set) : Set where
    -infty :            [ A ]
    emb_   : (a : A) -> [ A ]
    +infty :            [ A ]
  \end{minted}

  Define a type of ordered lists:
  \begin{minted}{haskell}
  data USL' (lb ub : [ A ]) : Set where
    []     : lb < ub -> USL' lb ub
    _,_::_ : hd -> lb < emb hd -> USL' (emb hd) ub ->
             USL' lb ub
  \end{minted}

  Top level type: relax the bounds as much as possible!
  \begin{minted}{haskell}
  type USL A = USL' (-infty : [ A ]) +infty
  \end{minted}
\end{frame}

\begin{frame}[fragile]{CLI}

\begin{minted}{haskell}
data Modifier name where
  mkFlag    : Record _ Flag   -> Modifier name
  mkOption  : Record _ Option -> Modifier name

record Command name : Set where
  inductive; constructor mkCommand
  field
    description : String
    subcommands : names ** Record names Command
    modifiers   : names ** Record names Modifier
    arguments   : Arguments
\end{minted}
\end{frame}

\section{Design a nice interface}
\begin{frame}[fragile]{We can \textbf{run} an awful lot at \textbf{compile time}}

  Fully-explicit, invariant-heavy structures internally

  vs. Decidability on concrete instances externally (smart constructors)

  Remember:
  \begin{minted}{haskell}
    , "-v" ::= flag "Invert match"
    < "-i" ::= flag "Ignore case"
    < "-e" ::= option "Regexp" regexp
    < <>
  \end{minted}

  \onslide<2>
  Using the smart constructors:
  \begin{minted}{haskell}
    <>      : Record [] _
    _::=_<_ : forall n -> S -> Record nms fields ->
              Record (insert n nms) (Finsert n nms S)
  \end{minted}
  
\end{frame}

\section{Generic Programming over Interfaces}

\begin{frame}[fragile]{Parsing}
  Parsing is decomposed in 3 phases

  \begin{itemize}
    \item subcommand selection
    \item modifier and arguments collection
    \item argument collection (triggered by ``-{}-'')
  \end{itemize}

  And the returned result is \emph{guaranteed} to respect the CLI:

  \begin{minted}{haskell}
parseCLI : (c : CLI) -> List String -> Error (ParsedCLI c)
withCLI  : (c : CLI) (k : ParsedCLI c -> IO a) -> IO a
  \end{minted}
\end{frame}

\begin{frame}[fragile]{Usage information}
We know a lot about the structure of the interface. Let's use it!
\begin{minted}{haskell}
usage : CLI -> String
\end{minted}

e.g.

\begin{minted}{bash}
grep       Print lines matching a regexp
  -e         Regexp
  -i         Ignore case
  -v         Invert match
\end{minted}
\end{frame}

\section{Conclusion}
\begin{frame}{Conclusion}
  \begin{itemize}
   \item Declarative
   \item Hierarchical
   \item Type-inference friendly
   \item Size-indexed internal representation
   \item Parser \& Usage
  \end{itemize}
\end{frame}

\begin{frame}[fragile]{Future Work}
  \begin{itemize}
    \item Validation DSL (cf. Jon Sterling's Vinyl)
    \item Syntactic sugar for writing the continuation
          \mintinline{haskell}{(k : ParsedCLI c -> IO a)}
    \item Compound flags
    \item Other types of documentation (e.g. man pages)
    \item More parsers for base types
    \item Set level issues
  \end{itemize}
\end{frame}
\end{document}
