\documentclass[dvipsnames]{beamer}
\usetheme{CambridgeUS}

\usepackage{minted}
\usepackage{mathspec}
\usepackage{xcolor}

\usepackage{tikz}
\usetikzlibrary{calc,shapes.multipart,chains,arrows}

\setmainfont[Ligatures=TeX]{XITS}
\setmathfont{XITS Math}

\title{agdARGS}
\subtitle{Command Line Arguments, Options and Flags}
\author[G. Allais]{Guillaume Allais}
\institute[Strathclyde]{University of Strathclyde}
\date[IDM, March 2015]{Idris Developers Meeting, March 2015}

\begin{document}

\begin{frame}
 \maketitle
\end{frame}

\section{Motivation}
\subsection{Let's compile!}

\begin{frame}
 \frametitle{Crazy thought: let's compile some programs!}

  Bored of the stereotype that in Type Theory we never go anywhere
  past typechecking or producing \texttt{.tex} documents (which,
  by the way, Travis can help with! \url{http://blog.gallais.org/travis-builds}).

  \vspace{0.7cm}

  Decided to build a simple executable because I can.
\end{frame}

\subsection{aGdaREP}

\begin{frame}{\url{https://github.com/gallais/aGdaREP}}

  My simple pick: a certified regexp matcher leading to an
  implementation of \texttt{grep}.

 \begin{center}
   \includegraphics[width=0.9\textwidth]{screenshot.png}
 \end{center}

  Lots of fun implementing, optimizing and extending the correct by
  construction matcher (see Alexandre Agular and Bassel Mannaa's 2009
  "Regular Expressions in Agda" ),

  But... then we need a user-facing interface!
\end{frame}

\begin{frame}[fragile]{A cheeky account of my journey}

  First step: add a binding to the important \texttt{Haskell}
  function...

\begin{minted}{haskell}
module Bindings.Arguments.Primitive where

open import IO.Primitive
open import Data.List
open import Data.String

{-# IMPORT System.Environment #-}

postulate
  getArgs : IO (List String)

{-# COMPILED getArgs System.Environment.getArgs #-}

\end{minted}
\end{frame}

\begin{frame}[fragile]{But wait! There's more!}

  Then lift the bound primitive to the \texttt{IO} type actually used
  at a high level of abstraction:

\begin{minted}{haskell}
module Bindings.Arguments where

open import Data.List
open import Data.String
open import IO
import Bindings.Arguments.Primitive as Prim

getArgs : IO (List String)
getArgs = lift Prim.getArgs
\end{minted}

  I guess it's a good way to learn about the language's and the
  standard library's internals? Which, maybe, I did not want to.
  Anyway.
\end{frame}

\begin{frame}[fragile]{"Hand-crafted" solution}

Now that we have access to the arguments, we just have to make sense
of them. We use a type of options:

\begin{minted}{haskell}
record grepOptions : Set where
  field
    -V     : Bool            -- version
    -v     : Bool            -- invert match
    -i     : Bool            -- ignore case
    regexp : Maybe String    -- regular expression
    files  : List FilePath   -- list of files to mine
open grepOptions public
\end{minted}

And "hand-craft" a function populating it:
\end{frame}

\begin{frame}[fragile]
\begin{minted}{haskell}
parseOptions : List String -> grepOptions
parseOptions args =
  record result { files = reverse (files result) }
  where
    cons : grepOptions -> String -> grepOptions
    cons opt "-v" = record opt { -v = true }
    cons opt "-V" = record opt { -V = true }
    cons opt "-i" = record opt { -i = true }
    cons opt str  =
      if is-nothing (regexp opt)
      then record opt { regexp = just str }
      else record opt { files  = str :: files opt }

    result : grepOptions
    result = foldl cons defaultGrepOptions args
\end{minted}
\end{frame}

\begin{frame}[fragile]{A few issues}
  \begin{itemize}
    \item I don't want to have to write this for every app
    \item I'm not even dealing with options yet
    \item This is not even ready for consumption yet!
      \begin{minted}{haskell}
       regexp : Maybe String
      \end{minted}
  \end{itemize}
\end{frame}

\section{Design the internals}
\subsection{Types}

\begin{frame}{What is a command-line interface?}

  \begin{itemize}
    \item A set of \emph{distinct} flag or options
    \item Each potentially coming with an \emph{argument}
    \item Living in a \emph{domain} of values
    \item We know how to \emph{parse}
  \end{itemize}
\end{frame}

\begin{frame}[fragile]{The (minimal) type of an \texttt{Argument}}
\begin{minted}{haskell}
record Argument (l : Level) : Set (suc l) where
  field
    flag        : String
    domain      : Domain l
    parser      : parserType domain

data Domain (l : Level) : Set (suc l) where
  None :                     Domain l
  Some : (S : Set l)      -> Domain l
  ALot : (M : RawMagma l) -> Domain l

parserType : {l : Level} -> Domain l -> Set l
parserType None     = Lift Unit
parserType (Some S) = String -> String || S
parserType (ALot M) = String -> String || carrier M
\end{minted}
\end{frame}

\begin{frame}[fragile]
  A CLI is defined by an \textbf{extensible record} of arguments.

  \begin{itemize}
    \item guaranteed uniqueness of flags
    \item easy to lookup values
    \item easy to extend
    \item first class citizens (generic programming possible!)
  \end{itemize}
\end{frame}

\subsection{Types - Keep your neighbours in order}

\begin{frame}{The type of extensible records}
  McBride to the rescue: "How to keep your neighbours in order" tells
  us how to build in the invariant stating that a tree's leaves are
  sorted.

  In the special case of linked lists, using a \emph{strict} total
  order, we move from:
%%%%%%%%%%%%%%%%%%%%%%%%%%%%%%%%%%%%%%%%%%%%%%%%%%%%%%%%%%%%
%%%%%%%%%%%%%%%%%%%%%%%%%%%%%%%%%%%%%%%%%%%%%%%%%%%%%%%%%%%%
\begin{figure}[t]
\centering
\begin{tikzpicture}[list/.style={rectangle split, rectangle split parts=2,
    draw, rectangle split horizontal}, >=stealth, start chain]

  \node[list,on chain] (A) {12};
  \node[list,on chain] (B) {99};
  \node[list,on chain] (C) {128};
  \node[on chain,draw, inner sep=6pt] (D) {};
  \draw (D.north west) -- (D.south east);
  \draw (D.north east) -- (D.south west);
  \draw[*->] let \p1 = (A.two), \p2 = (A.center) in (\x1,\y2) -- (B);
  \draw[*->] let \p1 = (B.two), \p2 = (B.center) in (\x1,\y2) -- (C);
  \draw[*->] let \p1 = (C.two), \p2 = (C.center) in (\x1,\y2) -- (D);
\end{tikzpicture}
\end{figure}

  To the proven ordered:

\begin{figure}[t]
\begin{tikzpicture}[list/.style={rectangle split, rectangle split parts=2,
    draw, rectangle split horizontal}, >=stealth, start chain]

  \node[list,on chain] (A) {{\color{red}-$\infty$} < 12};
  \node[list,on chain] (B) {{\color{gray}12} < 99};
  \node[list,on chain] (C) {{\color{gray}99} < 128};
  \node[on chain,draw] (D) {{\color{gray}128} < {\color{red}+$\infty$}};
  \draw[*->] let \p1 = (A.two), \p2 = (A.center) in (\x1,\y2) -- (B);
  \draw[*->] let \p1 = (B.two), \p2 = (B.center) in (\x1,\y2) -- (C);
  \draw[*->] let \p1 = (C.two), \p2 = (C.center) in (\x1,\y2) -- (D);
\end{tikzpicture}
\end{figure}

%%%%%%%%%%%%%%%%%%%%%%%%%%%%%%%%%%%%%%%%%%%%%%%%%%%%%%%%%%%%%%%%%%%%
%%%%%%%%%%%%%%%%%%%%%%%%%%%%%%%%%%%%%%%%%%%%%%%%%%%%%%%%%%%%%%%%%%%%
\end{frame}

\begin{frame}[fragile]{Key ideas}

  Extend any ordered set with +/-infinity:
  \begin{minted}{haskell}
  data [_] {l : Level} (A : Set l) : Set l where
    -infty :            [ A ]
    emb_   : (a : A) -> [ A ]
    +infty :            [ A ]
  \end{minted}

  Define a type of ordered lists:
  \begin{minted}{haskell}
  data USL (lb ub : Carrier) : Set _ where
    []     : lb < ub -> USL lb ub
    _,_::_ : hd (lt : lb < emb hd) (tl : USL (emb hd) ub) ->
             USL lb ub
  \end{minted}

  Top level type: relax the bounds as much as possible!
  \begin{minted}{haskell}
  Arguments = USL -infty +infty
  \end{minted}
\end{frame}

\begin{frame}[fragile]{}
  As a consequence:

  \begin{itemize}
    \item really easy to write the correct:
      {\small\mintinline{haskell}{insertM : lb < x < ub -> USL lb ub -> Maybe (USL lb ub)}}

    rather than the intuitive:
      {\small \mintinline{haskell}{insertM : x -> USL lb ub -> Maybe (USL (min lb x) (max x ub))}}

    \item We can search for values satisfying a decidable property:

      \mintinline{haskell}{search : (d : Decidable R) f (a : A) (xs : USL lb ub) ->}
      \mintinline{haskell}{Dec (el ** el inUSL xs ** R (f el) a))}
  \end{itemize}

  where \mintinline{haskell}{_inUSL_} are fancy de Bruijn indices:
  \begin{minted}{haskell}
  data _inUSL_ (a : _) : USL lb ub -> Set _ where
    z :               a inUSL a , lt :: xs
    s : a inUSL xs -> a inUSL b , lt :: xs
  \end{minted}
\end{frame}

\subsection{Terms}

\begin{frame}[fragile]{The values of type an extensible record}
  Jon Sterling to the rescue: "Vinyl: Modern Records for Haskell" tells
  us what they should look like.

\begin{minted}{haskell}
  Mode : (args : USL lb ub) -> Set (suc l)
  Mode args = arg (pr : arg inUSL args) -> Set l

  options : (args : USL lb ub) (m : Mode args) -> Set l
  options []                m = Lift Unit
  options (hd , lt :: args) m = m hd 0 * options args m'
\end{minted}
\end{frame}

\begin{frame}[fragile]{Benefits}
  \begin{itemize}
    \item Mode morphisms $\Rightarrow$ record morphisms
    \item \mintinline{haskell}{get} through \mintinline{haskell}{search} +
          \mintinline{haskell}{lookup : (pr : x inUSL args)}
          \mintinline{haskell}{(opts : options args m) -> m x pr}
    \item generic parsing function!
          \mintinline{haskell}{parse : List String -> (args : Arguments) ->}
          \mintinline{haskell}{String || options args MaybeMode}
    \item generic usage function!
          \mintinline{haskell}{usage : Arguments -> String}
  \end{itemize}

\end{frame}

\section{Design a nice interface}
\begin{frame}[fragile]{We can \textbf{run} an awful lot at \textbf{compile time}}

  \begin{itemize}
    \item Type-rich structures internally
    \item Decidability on concrete instances externally (smart constructors)
  \end{itemize}

  For instance using \texttt{fromJust}:

\begin{minted}{haskell}
fromJust : (a : Maybe A) {pr : maybe (\_ -> Unit) Void a}
           -> A
fromJust (just a) {pr} = a
fromJust nothing  {()}
\end{minted}

  we can turn \mintinline{haskell}{insertM : x (xs : Arguments) -> Maybe Arguments}
  into:

\begin{minted}{haskell}
insert : x (xs : Arguments) {pr : _} -> Arguments
insert x xs = fromJust (insertM x xs)
\end{minted}
\end{frame}

\section{Conclusion}

\begin{frame}{Future Work}
  \begin{itemize}
    \item Validation (DSL to write Mode morphisms?)
    \item Clean up the interface
    \item More parsers for base types
    \item Identify the utilities worth sending upstream
  \end{itemize}
\end{frame}

\end{document}
